\documentclass[11pt]{article}

% Use wide margins, but not quite so wide as fullpage.sty
\marginparwidth 0.5in 
\oddsidemargin 0.25in 
\evensidemargin 0.25in 
\marginparsep 0.25in
\topmargin 0.25in 
\textwidth 6in \textheight 8 in
% That's about enough definitions

% multirow allows you to combine rows in columns
\usepackage{multirow}
% tabularx allows manual tweaking of column width
\usepackage{tabularx}
% longtable does better format for tables that span pages
\usepackage{longtable}

\begin{document}
% this is an alternate method of creating a title
%\hfill\vbox{\hbox{Gius, Mark}
%       \hbox{Cpe 456, Section 01}  
%       \hbox{Lab 1}    
%       \hbox{\today}}\par
%
%\bigskip
%\centerline{\Large\bf Lab 1: Security Audit}\par
%\bigskip
\author{William Almeida Robert}
\title{Verification of Models in C ++ 11  Using Formal Methods}
\maketitle

\section{Summary}

\begin{center} SOFTWARE VERIFICATION OF C++ PROGRAMS\end{center}

\setlength{\parindent}{10ex} Software reliability is an important concern in the modern world. Due to the greater dependence of computer systems, its efficiency and safety are fundamental concerns to avoid accidents and damages in our daily lives. In order to achieve the reliability of these programs, software verifiers demonstrate to be a powerful tool to test its features. For this application, we are using the ESBMC++ software verifier, developed by the System and Software Verification Lab at the Federal University of Amazonas. \par 

\setlength{\parindent}{10ex}This tool has an automatic and efficient approach to check programs written on C++ programming language.
	The operation of ESBMC++ works on model checking, which is the process of exhaustively and automatically check proprieties of a system according to a given specification. In this case, it is particularly hard to handle C++ programs due to the complex language features, such as inheritance and polymorphism. To overcome these problems, ESBMC ++ uses an operational model, a simplified abstraction of the standard C++ libraries representing the classes, methods, and other features similar to the actual structure.\par 
\setlength{\parindent}{10ex}The main challenge of the verification process is to handle large programs and to support the complex features that the languages offers. The development of a comprehensive and reliable operational model is important to guarantee the efficiency of the tool along with the use of satisfiability modulo theories (SMT) to guide the checking process.\par 

\setlength{\parindent}{10ex}The development of the ESBMC++ was made from an integration with an already existing model checker, the ESBMC. As results of experiments with most of the features of C++, the tool demonstrated to be able handle inheritances, polymorphisms and exception handling.\par 

\setlength{\parindent}{10ex}The improvement of software verifiers such as ESBMC++ can have a huge impact on software development. Such tools could avoid errors, prevent fails and accidents in industries, transport systems, companies, saving money and people’s lives. The next step on this research is to give the software verifier the ability to check programs built on newer versions of C++, such as the C++ 11. This version contains new features on the standard library such as “arrays”, “forward list” and other libraries that need to be implemented on ESBMC++ as new operational models.\par 


\section{Activities}

\begin{itemize}
	\item Initial date: 29/04/2017 -- Final date: 29/05/2017
	\begin{enumerate}
		\item Write a Lay Summary {\it (done)}
		\item Study the operation of GitHub {\it (done)}
		\item Create a public repository in GitHub for the C++ operating model {\it (done)}
		\item Write a README for the project {\it (done)}
		\item Rewrite PIBIC proposal {\it (done)}
	\end{enumerate}
	
	\item Initial date: 29/05/2017 -- Final date: 05/06/2017
	\begin{enumerate}
		\item Build a UML diagram {\it (failed)}
		\item Rewrite the Lay Summary {\it (done)}
		\item Rewrite the README {\it (done)}		
	\end{enumerate}
	
	\item Initial date: 05/06/2017 -- Final date: 12/06/2017
	\begin{enumerate}
		\item Finish the UML diagram {\it (done)}
		\item Read articles about ESBMC++ {\it (done)}				
	\end{enumerate}
	
	\item Initial date: 12/06/2017 -- Final date: 19/06/2017
	\begin{enumerate}
		\item Build a hierarchy diagram of C++ libraries {\it (done)}
		\item Add Improvements to UML Diagrams {\it (done)}
		    \subitem Include captions;
		    \subitem Add missing attributes of some classes;
		    \subitem Identify classes, methods and implementations of C++11;
	\end{enumerate}
	
		\item Initial date: 19/06/2017 -- Final date: 26/06/2017
	\begin{enumerate}
		\item Add a doc/ directory to the repository with the UML and hierarchy diagram {\it (done)}
		\item Read the Text "How to create and review a GitHub pull request" and follow the according process to Pull Request on Git repository {\it (done)}
		\item Do a Regression Test on C++ Containers Classes {\it (done)}
			\subitem Find the codes;
			\subitem Perform the tests;
			\subitem Assertives;
			\subitem Build Failed codes;
			\subitem Test the fail codes;
	\end{enumerate}
	
	\item Initial date: 26/06/2017 -- Final date: 03/06/2017
	\begin{enumerate}
		\item Build the regression for the container classes: {\it (doing)}
		    \subitem Array;
		    \subitem Vector;
		    \subitem Dequeue;
		
	\end{enumerate}
	
	
\end{itemize}

\end{document}
